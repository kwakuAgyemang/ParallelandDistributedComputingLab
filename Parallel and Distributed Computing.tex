\documentclass{article}
\usepackage[utf8]{inputenc}
\usepackage{geometry}
\geometry{a4paper, portrait, margin=1in}

\title{\textbf{CS434:Parallel and Distributed Computing}}


\author{[Kwaku Ofosu-Agyeman]}
\date{[02/11/21]}


\begin{document}

\maketitle

\begin{center}  
\begin{large}  
\textbf{ Lab 1\\}
\end{large} 
\end{center} 

\newpage

World Ranking \\
Rank  1  :  Supercomputer Fugaku \\
Location : Japan \\
URL: https://www.r-ccs.riken.jp/en/fugaku/project \\
Manufacturer: Fujitsu \\
Memory: 5,087,232 gigabytes \\
Number of cores: 7,630,848 \\
Processor Type: A64FX 48C 2.2GHz \\
Interconnect:  Tofu Interconnect D \\
Linpack Performance: 442,010 TFlop/s\\
Theoretical Peak: 537,212 TFlop/s\\
Power Consumption:  29,899.23 kW (Optimized: 26248.36 kW)\\
Operating System: Red Hat Enterprise Linux\\
Interesting Features: Math Library, Compiler and MPI: FUJITSU Software Technical Computing Suite V4.0\\

Rank \ 2 \ : \ Summit-IBM PowerSystem \\
Location : United States\\
URL:  http://www.olcf.ornl.gov/olcf-resources/compute-systems/summit/\\
Manufacturer: IBM\\
Memory: 2,801,664 GB\\
Number of cores: 2,414,592\\
 Processor Type: IBM POWER9 22C 3.07GHz\\
Interconnect: Dual-rail Mellanox EDR Infiniband\\
Linpack Performance: 148,600 TFlop/s\\
Theoretical Peak: 200,795 TFlop/s\\
Power Consumption: 10,096.00 kW (Submitted)\\
Operating System: RHEL 7.4\\
Interesting Features: SPECTRUM MPI, XLC, nvcc\\

\newpage

Rank \ 3 \ : \ Sierra-IBM Power System \\
Location : United States\\
URL:  https://hpc.llnl.gov/hardware/platforms/sierra\\
Manufacturer: IBM / NVIDIA / Mellanox\\
Memory: 1,382,400 GB\\
Number of cores: 1,572,480\\
 Processor Type: IBM POWER9 22C 3.1GHz\\
Interconnect: Dual-rail Mellanox EDR Infiniband\\
Linpack Performance: 94,640 TFlop/s\\
Theoretical Peak: 125,712 TFlop/s\\
Power Consumption: 7,438.28 kW (Submitted) \\
Operating System: Red Hat Enterprise Linux\\
Interesting Features: IBM XLC ESSL, CUBLAS 9.2 IBM Spectrum MPI\\

Rank \ 4 \ : \ Sunway TaihuLight \\
Location : China\\
Manufacturer: NRCPC\\
Memory: 1,310,720 GB\\
Number of cores: 10,649,600\\
 Processor Type: Sunway SW26010 260C 1.45GHz\\
Interconnect: Sunway\\
Linpack Performance: 93,014.6 TFlop/s\\
Theoretical Peak: 125,436 TFlop/s\\
Power Consumption: 15,371.00 kW (Submitted)\\
Operating System: Sunway RaiseOS 2.0.5\\
Interesting Features: Nmax: 12,288,000, HPCG: 480.848\\
\newpage

Rank \ 5 \ : \ Selene \\
Location : United States\\
URL: https://www.nvidia.com/DGXSuperPOD\\
Manufacturer: Nvidia\\
Memory: 1,120,000 GB\\
Number of cores: 555,520\\
 Processor Type: AMD EPYC 7742 64C 2.25GHz\\
Interconnect: Mellanox HDR Infiniband\\
Linpack Performance: 63,460 TFlop/s\\
Theoretical Peak: 79,215 TFlop/s\\
Power Consumption: 2,646.00 kW (Submitted)\\
Operating System: Ubuntu 20.04.1 LTS\\
Interesting Features: Compiler: NVIDIA NVCC V11, Intel, Composer 2020.0.166, MPI: OpenMPI 4.0.3\\

\newpage
Africa Ranking \\
Rank \ 1 \ : \ Toubkal \\
Location : Morocco\\
URL:  hhttps://www.ascc.um6p.ma\\
Manufacturer: Dell EMC\\
Memory: 244,224 GB\\
Number of cores: 71,232\\
 Processor Type: Xeon Platnium 8276L 28C 2.2GHz\\
Interconnect: Mellanox InfiniBand HDR100\\
Linpack Performance: 3,158.11 TFlop/s\\
Theoretical Peak: 5,014.73 TFlop/s\\
Operating System: CentOS Scientific-OpenStack\\
Interesting Features: Compiler:  Intel, MPI: Intel MPI 2020.2\\

To incorporate the use of Linux on my Laptop, I decided to dual boot it and install Ubuntu 20.0.4 LTS on my machine. In installing it I faced a few challenges like switching my SATA configuration to AHCI since I could not run Ubuntu with that configuration. Because of this, each time I boot my laptop I have to switch between the AHCI configuration and the default intel SATA configuration depending on whether I want to boot with Windows or Ubuntu. I also had to do some additional installations after installing Visual Studio code to since the terminal could not identify the gcc command.  After that I was able to compile and run the code smoothly and successfully.


\end{document}